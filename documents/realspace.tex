% This project is part of the CoMiBaRaAn project.
% Copyright 2015 David W. Hogg (NYU)

% to-do
% -----
% - merge into some existing github repo

\documentclass[12pt]{article}
\input{vc}

\begin{document}

\section*{Real-space analysis of the \\ cosmic background radiation anisotropy}
\noindent
DWH and friends\\
{\footnotesize
% \textsl{Center for Cosmology and Particle Physics, Department of Physics, New York University}\\
% \textsl{Center for Data Science, New York University}\\
\texttt{draft document / git hash \githash\ (\gitdate)}}

\begin{abstract}
The cosmic microwave background (CMB) is almost always analyzed
in Fourier space ($\ell$--$m$ space)
rather than real space ($\theta$--$\phi$ space), and for good reason:
In Fourier space the (isotropic) physical cosmology generates a
diagonal variance tensor that depends only on $\ell$.
However, there are a number of important reasons to consider
real-space analyses:
The measurement noise variance tensor is (usually) diagonal in real space,
data defects and point sources are localized in real space but not in Fourier space,
the cut sky is in real space, the doppler boost of the CMB
is a real-space distortion (especially at high $\ell$), and the
Milky-Way foregrounds have non-trivial real-space templates.
Here we simultaneously present the real-space version of the standard
CMB analysis an also a likelihood function for probabilistic
inference.
We show that with new linear-algebra technology, the likelihood
function may be tractable, even for CMB maps with many millions or
billions of pixels.
\end{abstract}

Hello World!

\clearpage

Some notes to self:
\begin{itemize}
\item We should explicitly give (and plot) the angular kernel in real space corresponding to the standard $\Lambda$CDM model prediction.
\item There will be a point-source model and a Milky Way model added in to the CMB model.
\item The Doppler boost and the S-Z effect and the lensing all come in slightly differently from additively.
\item De-boosting is a great problem.  So is de-lensing.
\item There are frequency dependences and spatial structure; the model should capitalize on both.
\item If we can write things explicitly in the time domain, we get extra powers, not limited to the following:  We can isolate data defects and model them in a compact way.  We can treat stochastic (or regular) drifts in the instrumentation or calibration.
\item In principle the HODLR methodology should make things fast, but a bit issue is the ordering of the rows and columns of the matrices.
\end{itemize}

\end{document}
