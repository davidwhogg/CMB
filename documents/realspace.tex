% This project is part of the CMB project.
% Copyright 2015 David W. Hogg (NYU)

% to-do
% -----
% - merge into some existing github repo

\documentclass[12pt]{article}
\input{vc}

\begin{document}

\section*{Real-space analysis of the \\ cosmic background radiation anisotropy}
\noindent
David W. Hogg\\
{\footnotesize
\textsl{Center for Cosmology and Particle Physics, Department of Physics, New York University}\\
\textsl{Center for Data Science, New York University}\\
\texttt{draft document / git hash \githash\ (\gitdate)}}

\begin{abstract}
The cosmic microwave background (CMB) is almost always analyzed
in Fourier space ($\ell$--$m$ space)
rather than real space ($\theta$--$\phi$ space), and for good reason:
In Fourier space the (isotropic) physical cosmology generates a
diagonal variance tensor that depends only on $\ell$.
However, there are a number of important reasons to consider
real-space analyses:
The measurement noise variance tensor is (usually) diagonal in real space,
data defects and point sources are localized in real space but not in Fourier space,
the cut sky is in real space, the doppler boost of the CMB
is a real-space distortion (especially at high $\ell$), and the
Milky-Way foregrounds have non-trivial real-space templates.
Here we simultaneously present the real-space version of the standard
CMB analysis an also a likelihood function for probabilistic
inference.
We show that with new linear-algebra technology, the likelihood
function may be tractable, even for CMB maps with many millions or
billions of pixels.
\end{abstract}

Hello World!

\end{document}
