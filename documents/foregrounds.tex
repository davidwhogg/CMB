% This project is part of the CoMiBaRaAn project.
% Copyright 2015 David W. Hogg (NYU)

% to-do
% -----
% - merge into some existing github repo

\documentclass[12pt]{article}
\input{vc}

\begin{document}

\section*{A flexible, data-driven model for foregrounds \\ to the cosmic background radiation}
\noindent
DWH and friends\\
{\footnotesize
%\textsl{Center for Cosmology and Particle Physics, Department of Physics, New York University}\\
%\textsl{Center for Data Science, New York University}\\
\texttt{draft document / git hash \githash\ (\gitdate)}}

\begin{abstract}
Foreground emission from the Milky Way is a substantial contaminant to
almost all Cosmic Microwave Background (CMB) experiments.
The standard approach to foreground modeling (is this true?) is to
approximate the Milky Way as a linear sum of components with different
spectral energy distributions and to ``subtract it out'' in advance of
cosmological inference.
Here we develop and recommend a more general approach, in which the
foregrounds are generated by a non-linear function of $K$ components
or maps, where the non-linear function is given great freedom but
constrained to be a smooth function of wavelength.
The model is probabilistic, so it can be constrained simultaneously
with the cosmological parameters; it is a generalization of the
Gaussian process latent variable model (GPLVM).
\end{abstract}

Hello World!

\clearpage

Some notes to self:
\begin{itemize}
\item There are no preferred bands---that is, you don't want to model
  frequency X with some linear (or non-linear) combination of
  frequencies Y, Z, and W.  You want to model \emph{all} frequencies
  as a non-linear combination of lantent variables, which encode some
  combination of dust density, dust temperature, magnetic fields, and
  radiation fields.
\item The model can be at high resolution and make lower-resolution
  predictions; the model will be ``super-resolution'' in this sense.
\item There is a question of what basis to work in: It is not obvious
  that the real-space delta-function pixel basis is the best basis.
\item How to make the GP we use a smooth function of wavelength?  This
  is either trivial and dumb-ass or else almost impossible.
\end{itemize}

\end{document}
